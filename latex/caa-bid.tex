\documentclass[a4paper]{article}
%\usepackage{simplemargins}

\usepackage[
	pdftitle={CAA SIG Data Dragon},
	pdfsubject={CAA SIG Data Dragon},
	pdfauthor={M. Trognitz, E. Gruber, F. Thiery, K. Tolle and D. Wigg-Wolf},
	pdfkeywords={CAA}
]{hyperref}

% DOI and ARXIV Commands for Bib Files
% Written by Daniel Herber
% -----------------------------------------------
% one option is to use the 'note' field with this command
% -----------------------------------------------
% for example, if your doi is 10.2514/1.J052182
% then for the citation for the reference in your bib file, use
% note = "\doi{10.2514/1.J052182}",
% -----------------------------------------------
% for example, if your arxiv number is 0706.1234
% then for the citation for the reference in your bib file, use
% note = "\arxiv{0706.1234}",

% requires hyperref package for \href command
\usepackage{hyperref}

% doi command (use in bib file)
\newcommand{\doi}[1]{{doi:~\href{http://doi.org/#1}{#1}}\rmFullStop}

% arXiv command (use in bib file)
\newcommand{\arxiv}[1]{{arXiv:\href{https://arxiv.org/abs/#1}{#1}}\rmFullStop}

% command to remove full stop if the next character
\newcommand*{\rmFullStop}{\rmifnextchar{.}{}{}}

% command to check the next character and replace if present
% \rmifnextchar{X}{[removed text]}{[no X text]}
% if X is the next character, then it is removed and [removed text] is inserted
% otherwise, the character is not removed and [no X text] is inserted
% based on http://tex.stackexchange.com/questions/72827
\makeatletter
\newcommand{\rmifnextchar}[3]{%
  \begingroup
  \ltx@LocToksA{\endgroup#2}%
  \ltx@LocToksB{\endgroup#3}%
  \ltx@ifnextchar{#1}{%
    \def\next{\the\ltx@LocToksA}%
    \afterassignment\next
    \let\scratch= %
  }{%
    \the\ltx@LocToksB
  }%
}
\makeatother
%\RequirePackage{doi}
%\usepackage[square]{natbib}
\usepackage{amsmath}
\usepackage{amsfonts}
\usepackage{amssymb}
\usepackage{graphicx}

\begin{document}
\pagenumbering{gobble}

\normalsize

Dear honourable members of the CAA Steering Committee,
\newline


With this letter, we would like to express our intention to establish a CAA Special Interest Group (SIG) on \texttt{Semantics\ and\ LOUD\ in\ Archaeology} (CAA SIG Data Dragon) under the auspices of CAA international. The idea for this SIG was developed during the CAA conference 2019 in Krakow in a session on Linked Open Data (LOD) in archaeology. Out of this event and the resulting discussions, it was proposed to form a Special Interest Group in order to create a permanent platform for exchange and discussion of ideas and to develop the practical use of LOD for archaeological applications. During further discussions with colleagues, it became clear that it would be a good idea to extend the focus to related topics such as semantic modelling and the development and use of (programming) application interfaces for LOD, thus making them not only linked and open, but also usable, i.e.~LOUD.
\newline



The SIG de facto already exists with it's own website (http://datadragon.link), a GitHub organisation (https://github.com/caa-datadragons) and active members. For the CAA conference 2020, we would now like to propose to formally include this SIG in the canon of the SIGs of CAA. We have also organised a session on Linked (Open) Data where prospective members of the SIG can register.

\section{Hic sunt dracones}\label{hic-sunt-dracones}

In historical maps, the phrase \texttt{Hic\ sunt\ dracones} (engl. here be dragons) is used to describe areas which were unknown to the cartographer. Today the WWW gives researchers the possibility of sharing their research (data) and enables the community to participate in scientific discourse in order to create previously unknown knowledge. But much of this shared data are not findable or accessible, thus resulting in modern unknown data dragons. Often these data dragons lack connections to other datasets, i.e.~they are not interoperable and in some cases even lack usefulness or usability. To overcome these shortcomings, a set of techniques, standards and recommendations can be used: Semantic Web and Linked Open Data, the FAIR principles and LOUD data.

\section{LOUD and FAIR data in the context of CAA}\label{loud-and-fair-data-in-the-context-of-caa}

Various researchers are modelling semantic information and producing Linked Data and LOUD according to the FAIR principles in many archaeological data applications. Prominent forerunners in this field are the Numismatics Community (Ethan Gruber, David Wigg-Wolf, Karsten Tolle, \ldots{}) and the Pelagios Commons Community (Valeria Vitale, Leif Isaksen, \ldots{}). However, the work of other researchers (Florian Thiery, Martina Trognitz, \ldots{}), and previous talks at earlier CAA conferences, show that this field is being worked on widely beyond these specialist fields. We would like to further establish Linked Data in archaeology, enable beginners to use and produce Linked Data, invite other scientists for discussion, and embed LOD as an important topic through an SIG at CAA.

\section{Statement of purpose}\label{statement-of-purpose}

The core aim of the CAA Special Interest Group on \texttt{Semantics\ and\ LOUD\ in\ Archaeology} (CAA SIG Data Dragon) is to use CAA's SIG format for raising awareness of Linked (Open) Data in archaeology. This can be achieved through some primary objectives:

\begin{itemize}
\item
  Create a friendly and open platform to discuss the role of LOUD and FAIR Data in archaeology.
\item
  Enable the CAA community to learn the LOD basics through workshops, teaching materials, etc.
\item
  Tackle the challenge of developing LOUD publishing workflows in archaeology.
\item
  Integrate the semantic modelling \& ontology communities (e.g.~CIDOC CRM, SKOS, \ldots{}), as well as infrastructure communities (Pelagios, Nomisma, Wikidata, \ldots{}) into LOD in archaeology.
\item
  Forge closer collaboration between researchers already active in this field and allow for the education of other interested parties.
\item
  Provide a platform to present and evaluate the various venues which are open to archaeology and LOD.
\item
  Encourage development of related technologies, frameworks and theoretical perspectives.
\item
  Highlight and anchor the topic of data quality in LOUD publishing.
\item
  Strengthen the subject of LOD gazetteers for research in archaeology and on the ancient world
\item
  Strengthen the subject of LOD thesauri, e.g.~Getty, Heritage England, \ldots{}
\item
  Built up a connection to the Wikidata ecosystem to extend beyond the scope of archaeology, benefit from public contributions, and feed LOD back into Wikidata.
\end{itemize}

This SIG is particularly supportive of students and early career researches who may be interested in developing their skills and promoting more pervasive use of Linked Data in semantics in research, publication and teaching.

The SIG aims at an open format that supports discussion and the circulation of ideas drawn from various perspectives. Everyone may participate on equal terms, following CAA International's established Ethics Policy. To ensure expedient and civil discourse, the SIG committee will remove, edit, or reject comments, commits, code, issues, and other contributions that are not aligned with this policy. To be as inclusive as possible, and also for the sake of reproducibility, we strongly prefer open-source over proprietary software.

The group further respects CAA's rules which govern special interest groups as detailed here: http://caa-international.org/special-interest-groups/.

\section{Related Activities at CAA}\label{related-activities-at-caa}

\begin{itemize}
\item
  Session on Ontologies at CAA Paris 2014 (S07 Ontologies and standards for improving interoperability of archaeological data: from models towards practical experiences in various contexts)
\item
  Session on Linked Data at CAA Siena 2015 (S2D Linked Data: From interoperable to interoperating)
\item
  Roundtable on Linked Data and Pottery at CAA Siena 2015 (RT5 Linked Open Data Applied to Pottery Databases)
\item
  Session on Linked Data at CAA Oslo 2016 (S21 Linked Pasts: Connecting islands of content)
\item
  Session on LOD and Data Quality at CAA Tübingen 2018 (S33 Guaranteeing data quality in archaeological Linked Open Data)
\item
  Session on LOD and Data Quality at CAA Krakow 2019 (S14 Modelling Data Quality in archaeological Linked Open Data)
\item
  Session on Linked Open and FAIR data at CAA Oxford 2020 (S21 Hic sunt dracones -- Improving knowledge exchange in the Semantic Web with Linked Open and FAIR data)
\end{itemize}

\section{Proposed SIG Activities at CAA}\label{proposed-sig-activities-at-caa}

\begin{itemize}
\item
  SIG Roundtable at the next CAAs on LOD in archaeology
\item
  SIG Session at the next CAAs on Semantics, LOUD and FAIR data in archaeology
\item
  Introductory workshops at upcoming CAAs to teach basic LOD and FAIR skills
\end{itemize}

\section{SIG Outreach}\label{sig-outreach}

\begin{itemize}
\item
  The SIG aims to make the \textbf{Data Dragons} visible!
\item
  Website \texttt{http://datadragon.link} to share all information on SIG activities and on the topic of LOUD and FAIR itself
\item
  GitHub organisation \texttt{@caa-datadragons} with SIG members as owner; all others can do pull requests
\item
  Publish a list of related infrastructures for LOD in archaeology with their URI ressources, SPARQL enpoints, APIs, dumps, etc. for Objects (Nomisma etc.), Gazetteers (Pleiades etc.), Thesauri (Getty etc.), Time (ChronOntology etc.), Persons (GND etc.), GeoData (OpenStreeMap etc.) etc.
\item
  this is done through the \texttt{hungry\ squirrels} (\texttt{http://hungry.squirrel.link}) project and published in the \texttt{dragonator} (\texttt{http://dragonator.datadragon.link})
\end{itemize}

\section{Further SIG Outreach Ideas}\label{further-sig-outreach-ideas}

\begin{itemize}
\item
  Publish a \texttt{LOD\ in\ Archaeology\ Cookbook\ for\ Beginners}
\item
  Publish a \texttt{List\ of\ talks\ and\ papers\ related\ to\ LOD\ in\ archaeology} (by using Wikidata and Scholia)
\end{itemize}

\section{SIG Governance}\label{sig-governance}

The group was proposed at CAA 2019 in Krakow, it will be formed if consent is given by the CAA steering committee and it is ratified by the CAA membership at the Annual General Meeting at CAA 2020 in Oxford. Within a year the SIG will make arrangements to appoint a speaker. This person will cater for the day-to-day running of the SIG, the SIG committee will be formed of those currently active in this field of interest. This will be subject to the CAA steering committee's approval.

\section{SIG Coordinators \& Supporters}\label{sig-coordinators-supporters}

The group coordinators are formed of those who are demonstratively active in this arena, while all CAA members are welcome to contribute to the group.

\begin{itemize}
	\item temp. speaker
	\begin{itemize}
		\item Florian Thiery
	\end{itemize}
	\item temp. coordinators
	\begin{itemize}
		\item Martina Trognitz, Ethan Gruber
		\item Florian Thiery, David Wigg-Wolf
	\end{itemize}
	\item Members and Supporters
	\begin{itemize}
		\item \texttt{https://t1p.de/DDJOIN}
	\end{itemize}
\end{itemize}


If you approve the formation of this SIG, we would also like to ask you if and how the SIG could get a short slot in the schedule of the CAA2020 conference in Oxford to discuss the organisational affairs among new and established members.
\newline


Thank you for considering this proposal.
\newline


Sincerely yours
\newline


M. Trognitz, E. Gruber, F. Thiery, K. Tolle and D. Wigg-Wolf


\end{document}